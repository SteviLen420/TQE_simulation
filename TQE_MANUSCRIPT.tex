# **Theory of the Question of Existence (TQE):**

**An Energy–Information Coupling Hypothesis for the Stabilization of Physical Law**

### **Abstract**

Why do stable, complexity-permitting physical laws exist at all? I propose the Theory of the Question of Existence (TQE), a quantitative framework where such stability emerges from the coupling of vacuum energy fluctuations with an information-theoretic orientation parameter. I define this parameter via Kullback–Leibler divergence. In complementary analyses, I also employ Shannon entropy as an alternative measure of informational asymmetry. This parameter biases quantum state reduction toward law-consistent outcomes. Numerical simulations demonstrate that universes stabilize only within a narrow energetic “Goldilocks window,” where probability weights lock in and complexity becomes possible. Crucially, the model yields falsifiable predictions: it implies non-random statistical features in large-scale anomalies of the cosmic microwave background, including features such as the low quadrupole and hemispherical power asymmetry. TQE thus reframes the fine-tuning problem, presenting a potential mechanism for the dynamic selection of physical law at the origin of cosmogenesis.

### **Core Message**

I hypothesize that stable physical laws may arise from the coupling of energy fluctuations with an informational orientation. Energy alone can generate universes, but such systems are likely to remain in stable chaos without producing structure or complexity. If even a minimal informational bias is present, however, it may allow physical laws to lock in, thereby opening the path toward order, self-organization, and complexity. This framework is still at an early stage, and further research is required to clarify the role of informational orientation and to test the model’s predictions against empirical data.

### **1. Introduction**

The foundational premise of this framework addresses the origin of the universe itself, proposing it arises from a quantum fluctuation out of a state devoid of pre-existing physical laws. The mechanism for such an event is rooted in the Heisenberg Uncertainty Principle, which allows for the spontaneous emergence of energy from the vacuum. It is postulated that in a true pre-law state, such quantum fluctuations are not constrained by the established physics of a mature universe. Unbound by stable laws that would otherwise govern their scale, a fluctuation of sufficient magnitude could emerge, capable of initiating the cosmogenesis process described by the TQE model.

This hypothesis also offers a natural explanation for why new universes do not form within our own: the stable laws and constants that now govern our cosmos effectively suppress vacuum fluctuations, preventing them from reaching the universe-spawning magnitude required. Thus, a universe can be understood as an exceptionally rare, large-scale fluctuation from a "nothingness" governed only by fundamental quantum indeterminacy.

The persistence of stable physical laws enabling complexity is one of the central open questions in cosmology. Standard quantum field theory describes vacuum fluctuations but does not explain how law-governed universes emerge from them. Inflationary models address fine-tuning and expansion dynamics, while anthropic reasoning justifies observed conditions retrospectively, yet neither explains why particular configurations are preferentially realized.

I propose that stability originates from an intrinsic information orientation parameter, which introduces a systematic bias in the collapse of quantum superpositions. Coupled with vacuum energy fluctuations, this parameter generates a probabilistic mechanism for the emergence of physical law. Unlike anthropic or purely inflationary accounts, this framework offers a quantitative formulation and identifies potential empirical tests.

### **1.1 Extended Introduction**

The model introduces an **information orientation parameter (I)**, defined operationally as a normalized asymmetry between successive probability distributions. Computed via Kullback–Leibler divergence, this parameter quantifies directional bias in quantum state evolution toward complexity-permitting outcomes. Unlike philosophical treatments of “information,” this definition is mathematically precise and can be applied in both simulation and, in principle, observational analysis.

When coupled with vacuum energy fluctuations, this parameter enables a probabilistic selection mechanism for physical laws. Stability arises only within a narrow energetic window, analogous to critical thresholds in condensed matter systems, while the information bias enhances the likelihood of collapse into law-consistent states. This reframes the fine-tuning problem: stable physical laws are not imposed but dynamically selected through energy–information interaction.

Importantly, the framework yields falsifiable implications. If correct, one expects non-random statistical features in cosmic microwave background anomalies, such as the low quadrupole or large-scale alignments. These anomalies serve not as direct explanations but as diagnostic proxies of stabilization dynamics, offering a pathway to empirical validation of the model.

### **2. Theoretical Framework**

The foundation of our model rests on three key concepts: the quantum state of the universe, energy fluctuations, and a previously unaddressed, intrinsic property of energy, which we term Information (I).

Information (I) as an Intrinsic Property of Energy: We postulate that energy possesses a fundamental, intrinsic property: a directionality towards complexity. We call this property Information (I). This is not an external field but an inseparable aspect of energy itself that carries the structure-forming potential.

Let the state of the universe be represented by a superpositional probability distribution P(ψ). During early cosmogenesis, we assume this distribution is not static but subject to modulation by vacuum energy fluctuations (E) and the aforementioned Information parameter (I):

**P’(ψ) = P(ψ) f(E, I)**

where:

**P(ψ)** is the baseline quantum probability distribution.

**E** is vacuum fluctuation energy, sampled from a heavy-tailed (lognormal) distribution.

**I** is the information parameter, defined below as normalized asymmetry or orientation (0 ≤ I ≤ 1).

**f(E, I)** is a fine-tuning function biasing outcomes toward stability.

**P′(ψ)** is the modulated distribution after energy–information coupling.

### **2.1. Explicit Form of the Fine-Tuning Function**

I use the functional form:

**f(E, I) = exp(−(E − E_c)² / (2σ²)) · (1 + α · I)**

where:

E_c is the critical energy around which universes stabilize,

σ controls the width of the stability window,

α quantifies the strength of the information bias.

This captures two assumptions:

**Energetic Goldilocks zone:** stability occurs only around E_c.

**Information bias:** I increases the likelihood of collapse into complexity-permitting states.

While Eq. (1) provides the analytical form of **f(E, I)**, the **Monte Carlo** implementation uses a stochastic approximation (**‘Goldilocks noise’)**, where noise amplitude scales with distance from the stability window. This probabilistic scheme captures the same underlying mechanism of stabilization.

### **2.2 Model Parameters**

In the TQE framework, the modulation factor *f(E, I)* is governed by a minimal set of parameters that encode the stability conditions for universes. These are not arbitrary fitting constants but structural elements of the model:

**E_c – critical energy:** the center of the Goldilocks stability window. Universes with energies near *E_c* can stabilize physical laws.

**σ – stability width:** determines the tolerance around *E_c*. Larger σ broadens the stability window, while smaller σ makes stabilization rarer.

**α – orientation bias strength:** quantifies the effect of informational orientation *I*. For α = 0, orientation is irrelevant; larger α increases the probability of complexity-permitting universes.

**Lock-in criterion:** operationally defined as stabilization when relative probability change satisfies ΔP / P < 0.005 over at least 6 consecutive epochs.

In the default configuration of the numerical experiments, the values are set to (E_c = 4.0, σ = 4.0, α = 0.8) to yield a clear stability window. These parameters can be varied to probe the model's robustness.

### **2.3. Definition of the Information Parameter (I)**

I here define **I** formally as an **information-theoretic asymmetry measure** rather than a metaphysical quantity.

Concretely, I is estimated via the **Kullback–Leibler (KL) divergence** between probability distributions at successive epochs:

I = Dₖₗ(Pₜ || Pₜ₊₁) / (1 + Dₖₗ(Pₜ || Pₜ₊₁))

ensuring 0 ≤ I ≤ 1.

Thus, **I acts as a proxy for directional bias** in quantum state evolution, computable in both simulation and (in principle) observational contexts. The numerical implementation, however, explores a richer definition by also computing the Shannon entropy (H) of the state. In the default configuration, these two measures are combined via a product fusion **(I = I_kl × I_shannon)**, creating a composite parameter that captures both informational asymmetry and intrinsic complexity. However, this definition should be regarded as a first operational step: the precise formalization and physical grounding of **I** remain open questions that require further theoretical and empirical investigation.

### **2.4. Stability Condition (Lock-In Criterion)**

I define **law stabilization (lock-in)** when the relative variation of the system's key parameters satisfies:

**ΔP / P < 0.005 over at least 6 consecutive epochs.**

This operational definition, directly implemented in the simulation's MASTER_CTRL configuration (REL_EPS_LOCKIN = 5e-3, CALM_STEPS_LOCKIN = 6), provides an objective and reproducible criterion for distinguishing universes that stabilize from those that remain in chaos.

### **2.5. Goldilocks Zone as Emergent Critical Points**

The thresholds E_c^{low} and E_c^{high} are not arbitrary, but act as **emergent critical points**, analogous to phase transitions in condensed matter (e.g., superconducting T_c). They mark the energetic window where law stabilization becomes possible.

Illustrative simulations support this interpretation. Both Kullback–Leibler divergence and Shannon entropy were tested independently as orientation measures, as well as in combined form. In all cases, stabilization appeared only within narrow energetic intervals, though the exact location and breadth of these “Goldilocks zones” varied between runs. This variability suggests that stabilization is not tied to a single fixed energy level but emerges as a critical region shaped by the interaction of fluctuations and orientation. These results should be regarded as preliminary; further work is required to characterize the stability windows in detail and to confront them with cosmological simulations and observational data.

### **2.6. Relation to CMB Anomalies**

I emphasize that I do not claim the TQE model *explains* anomalies like the **low quadrupole** or the **Axis of Evil**. Rather, such features are natural **diagnostic proxies** of law-stabilization dynamics.

If the TQE mechanism is correct, one expects a **non-random distribution of large-scale anomalies**, statistically testable against Planck and WMAP data. This makes the model, in principle, falsifiable.

### **2.7. Literature Context**

The TQE framework resonates with:

**Wheeler’s “it from bit”** (information as reality’s foundation),

**Zurek’s Quantum Darwinism** (selection of robust states),

**Tegmark’s Mathematical Universe Hypothesis**,

**Davies on emergent laws**,

**Smolin’s cosmological natural selection**.

It extends these by offering a **quantitative, information-theoretic stabilization mechanism**.

---

## **3. Simulation Framework and Results**

**Methods – Randomness and Scope**

The simulation framework is built on a strong foundation of reproducibility. While exploratory runs can be performed with SEED, all key experiments presented are governed by a SEED defined in the configuration. This ensures that the entire ensemble of universes can be reproduced exactly, which is critical for scientific validation, debugging, and peer review.

Author’s Note

I am not a professional physicist, but an independent enthusiast. This work is intended as an exploratory contribution rather than a finished theory. I apologize for any possible inaccuracies or oversights, my goal is simply to share an idea that may inspire further exploration, refinement, and development by the broader research community.


