\section{\texorpdfstring{\textbf{Theory of the Question of Existence
(TQE):}}{Theory of the Question of Existence (TQE):}}\label{theory-of-the-question-of-existence-tqe}

\subsubsection{An Energy--Information Coupling Hypothesis for the
Stabilization of Physical
Law}\label{an-energyinformation-coupling-hypothesis-for-the-stabilization-of-physical-law}

\textbf{Author: Stefan Len}

\subsubsection{\texorpdfstring{\textbf{Abstract}}{Abstract}}\label{abstract}

Why do stable, complexity-permitting physical laws exist at all? I
propose the Theory of the Question of Existence (TQE), a quantitative
framework where such stability emerges from the coupling of vacuum
energy fluctuations with an information-theoretic orientation parameter.
I define this parameter via Kullback--Leibler divergence. In
complementary analyses, I also employ Shannon entropy as an alternative
measure of informational asymmetry. This parameter biases quantum state
reduction toward law-consistent outcomes. Numerical simulations
demonstrate that universes stabilize only within a narrow energetic
``Goldilocks window,'' where probability weights lock in and complexity
becomes possible. Crucially, the model yields falsifiable predictions:
it implies non-random statistical features in large-scale anomalies of
the cosmic microwave background, including features such as the low
quadrupole and hemispherical power asymmetry. TQE thus reframes the
fine-tuning problem, presenting a potential mechanism for the dynamic
selection of physical law at the origin of cosmogenesis.

\subsubsection{\texorpdfstring{\textbf{Core
Message}}{Core Message}}\label{core-message}

I hypothesize that stable physical laws may arise from the coupling of
energy fluctuations with an informational orientation. Energy alone can
generate universes, but such systems are likely to remain in stable
chaos without producing structure or complexity. If even a minimal
informational bias is present, however, it may allow physical laws to
lock in, thereby opening the path toward order, self-organization, and
complexity. This framework is still at an early stage, and further
research is required to clarify the role of informational orientation
and to test the model's predictions against empirical data.

\subsubsection{\texorpdfstring{\textbf{1.
Introduction}}{1. Introduction}}\label{introduction}

The foundational premise of this framework addresses the origin of the
universe itself, proposing it arises from a quantum fluctuation out of a
state devoid of pre-existing physical laws. The mechanism for such an
event is rooted in the Heisenberg Uncertainty Principle, which allows
for the spontaneous emergence of energy from the vacuum. It is
postulated that in a true pre-law state, such quantum fluctuations are
not constrained by the established physics of a mature universe. Unbound
by stable laws that would otherwise govern their scale, a fluctuation of
sufficient magnitude could emerge, capable of initiating the
cosmogenesis process described by the TQE model.

This hypothesis also offers a natural explanation for why new universes
do not form within our own: the stable laws and constants that now
govern our cosmos effectively suppress vacuum fluctuations, preventing
them from reaching the universe-spawning magnitude required. Thus, a
universe can be understood as an exceptionally rare, large-scale
fluctuation from a ``nothingness'' governed only by fundamental quantum
indeterminacy.

The persistence of stable physical laws enabling complexity is one of
the central open questions in cosmology. Standard quantum field theory
describes vacuum fluctuations but does not explain how law-governed
universes emerge from them. Inflationary models address fine-tuning and
expansion dynamics, while anthropic reasoning justifies observed
conditions retrospectively, yet neither explains why particular
configurations are preferentially realized.

I propose that stability originates from an intrinsic information
orientation parameter, which introduces a systematic bias in the
collapse of quantum superpositions. Coupled with vacuum energy
fluctuations, this parameter generates a probabilistic mechanism for the
emergence of physical law. Unlike anthropic or purely inflationary
accounts, this framework offers a quantitative formulation and
identifies potential empirical tests.

\subsubsection{\texorpdfstring{\textbf{1.1 Extended
Introduction}}{1.1 Extended Introduction}}\label{extended-introduction}

The model introduces an \textbf{information orientation parameter}
\((I)\), defined operationally as a normalized asymmetry between
successive probability distributions. Computed via Kullback--Leibler
divergence, this parameter quantifies directional bias in quantum state
evolution toward complexity-permitting outcomes. Unlike philosophical
treatments of ``information,'' this definition is mathematically precise
and can be applied in both simulation and, in principle, observational
analysis.

When coupled with vacuum energy fluctuations, this parameter enables a
probabilistic selection mechanism for physical laws. Stability arises
only within a narrow energetic window, analogous to critical thresholds
in condensed matter systems, while the information bias enhances the
likelihood of collapse into law-consistent states. This reframes the
fine-tuning problem: stable physical laws are not imposed but
dynamically selected through energy--information interaction.

Importantly, the framework yields falsifiable implications. If correct,
one expects non-random statistical features in cosmic microwave
background anomalies, such as the low quadrupole or large-scale
alignments. These anomalies serve not as direct explanations but as
diagnostic proxies of stabilization dynamics, offering a pathway to
empirical validation of the model.

\subsubsection{\texorpdfstring{\textbf{2. Theoretical
Framework}}{2. Theoretical Framework}}\label{theoretical-framework}

The foundation of our model rests on three key concepts: the quantum
state of the universe, energy fluctuations, and a previously
unaddressed, intrinsic property of energy, which we term Information
\((I)\).

Information \((I)\) as an Intrinsic Property of Energy: We postulate
that energy possesses a fundamental, intrinsic property: a
directionality towards complexity. We call this property Information
\((I)\). This is not an external field but an inseparable aspect of
energy itself that carries the structure-forming potential.

Let the state of the universe be represented by a superpositional
probability distribution \(P(ψ)\). During early cosmogenesis, we assume
this distribution is not static but subject to modulation by vacuum
energy fluctuations \((E)\) and the aforementioned Information parameter
\((I)\):

\[
P′(ψ)=P(ψ)⋅f(E,I)
\]

where:

\(P(ψ)\) is the baseline quantum probability distribution.

\(E\) is vacuum fluctuation energy, sampled from a heavy-tailed
(lognormal) distribution.

\(I\) is the information parameter, defined below as normalized
asymmetry or orientation (0 ≤ I ≤ 1).

\(f(E, I)\) is a fine-tuning function biasing outcomes toward stability.

\(P′(ψ)\) is the modulated distribution after energy--information
coupling.

\subsubsection{\texorpdfstring{\textbf{2.1. Explicit Form of the
Fine-Tuning
Function}}{2.1. Explicit Form of the Fine-Tuning Function}}\label{explicit-form-of-the-fine-tuning-function}

I use the functional form:

\[
f(E, I) = exp(−(E − E_c)² / (2σ²)) · (1 + α · I)
\]

where:

\(E_c\) is the critical energy around which universes stabilize,

\(σ\) controls the width of the stability window,

\(α\) quantifies the strength of the information bias.

This captures two assumptions:

\textbf{Energetic Goldilocks zone:} stability occurs only around
\(E_c\).

\textbf{Information bias:} I increases the likelihood of collapse into
complexity-permitting states.

While Eq. (1) provides the analytical form of \(f(E, I)\), the
\textbf{Monte Carlo} implementation uses a stochastic approximation
(\textbf{`Goldilocks noise')}, where noise amplitude scales with
distance from the stability window. This probabilistic scheme captures
the same underlying mechanism of stabilization.

\subsubsection{\texorpdfstring{\textbf{2.2 Model
Parameters}}{2.2 Model Parameters}}\label{model-parameters}

In the TQE framework, the modulation factor \(f(E, I)\) is governed by a
minimal set of parameters that encode the stability conditions for
universes. These are not arbitrary fitting constants but structural
elements of the model:

\(E_c\) -- \textbf{critical energy}: the center of the Goldilocks
stability window. Universes with energies near \(E_c\) can stabilize
physical laws.

\(σ\) -- \textbf{stability width}: determines the tolerance around
\(E_c\). Larger σ broadens the stability window, while smaller σ makes
stabilization rarer.

\(α\) -- \textbf{orientation bias strength}: quantifies the effect of
informational orientation \(I\). For \(α = 0\), orientation is
irrelevant; larger α increases the probability of complexity-permitting
universes.

\textbf{Lock-in criterion:} operationally defined as stabilization when
relative probability change satisfies \(ΔP / P < 0.005\) over at least 6
consecutive epochs.

In the default configuration of the numerical experiments, the values
are set to \((E_c = 4.0, σ = 4.0, α = 0.8)\) to yield a clear stability
window. These parameters can be varied to probe the model's robustness.

\subsubsection{\texorpdfstring{\textbf{2.3. Definition of the
Information Parameter
(I)}}{2.3. Definition of the Information Parameter (I)}}\label{definition-of-the-information-parameter-i}

I here define \textbf{I} formally as an \textbf{information-theoretic
asymmetry measure} rather than a metaphysical quantity.

Concretely, I is estimated via the \textbf{Kullback--Leibler (KL)
divergence} between probability distributions at successive epochs:

\[
I = Dₖₗ(Pₜ || Pₜ₊₁) / (1 + Dₖₗ(Pₜ || Pₜ₊₁))
\]

ensuring \(0 ≤ I ≤ 1\).

Thus, \textbf{I acts as a proxy for directional bias} in quantum state
evolution, computable in both simulation and (in principle)
observational contexts. The numerical implementation, however, explores
a richer definition by also computing the Shannon entropy (H) of the
state. In the default configuration, these two measures are combined via
a product fusion \textbf{(I = I\_kl × I\_shannon)}, creating a composite
parameter that captures both informational asymmetry and intrinsic
complexity. However, this definition should be regarded as a first
operational step: the precise formalization and physical grounding of
\(I\) remain open questions that require further theoretical and
empirical investigation.

\subsubsection{\texorpdfstring{\textbf{2.4. Stability Condition (Lock-In
Criterion)}}{2.4. Stability Condition (Lock-In Criterion)}}\label{stability-condition-lock-in-criterion}

I define \textbf{law stabilization (lock-in)} when the relative
variation of the system's key parameters satisfies:

\textbf{ΔP / P \textless{} 0.005 over at least 6 consecutive epochs.}

This operational definition, directly implemented in the simulation's
MASTER\_CTRL configuration (REL\_EPS\_LOCKIN = 5e-3, CALM\_STEPS\_LOCKIN
= 6), provides an objective and reproducible criterion for
distinguishing universes that stabilize from those that remain in chaos.

\subsubsection{\texorpdfstring{\textbf{2.5. Goldilocks Zone as Emergent
Critical
Points}}{2.5. Goldilocks Zone as Emergent Critical Points}}\label{goldilocks-zone-as-emergent-critical-points}

The thresholds \(E_c^{low}\) and \(E_c^{high}\) are not arbitrary, but
act as \textbf{emergent critical points}, analogous to phase transitions
in condensed matter (e.g., superconducting T\_c). They mark the
energetic window where law stabilization becomes possible.

Illustrative simulations support this interpretation. Both
Kullback--Leibler divergence and Shannon entropy were tested
independently as orientation measures, as well as in combined form. In
all cases, stabilization appeared only within narrow energetic
intervals, though the exact location and breadth of these ``Goldilocks
zones'' varied between runs. This variability suggests that
stabilization is not tied to a single fixed energy level but emerges as
a critical region shaped by the interaction of fluctuations and
orientation. These results should be regarded as preliminary; further
work is required to characterize the stability windows in detail and to
confront them with cosmological simulations and observational data.

\subsubsection{\texorpdfstring{\textbf{2.6. Relation to CMB
Anomalies}}{2.6. Relation to CMB Anomalies}}\label{relation-to-cmb-anomalies}

I emphasize that I do not claim the TQE model \emph{explains} anomalies
like the \textbf{low quadrupole} or the \textbf{Axis of Evil}. Rather,
such features are natural \textbf{diagnostic proxies} of
law-stabilization dynamics.

If the TQE mechanism is correct, one expects a \textbf{non-random
distribution of large-scale anomalies}, statistically testable against
Planck and WMAP data. This makes the model, in principle, falsifiable.

\subsubsection{\texorpdfstring{\textbf{2.7. Literature
Context}}{2.7. Literature Context}}\label{literature-context}

The TQE framework resonates with:

\textbf{Wheeler's ``it from bit''} (information as reality's
foundation),

\textbf{Zurek's Quantum Darwinism} (selection of robust states),

\textbf{Tegmark's Mathematical Universe Hypothesis},

\textbf{Davies on emergent laws},

\textbf{Smolin's cosmological natural selection}.

It extends these by offering a \textbf{quantitative,
information-theoretic stabilization mechanism}.

\begin{center}\rule{0.5\linewidth}{0.5pt}\end{center}

\subsection{\texorpdfstring{\textbf{3. Simulation Framework and
Results}}{3. Simulation Framework and Results}}\label{simulation-framework-and-results}

\textbf{Methods -- Randomness and Scope}

The simulation framework is built on a strong foundation of
reproducibility. While exploratory runs can be performed with SEED, all
key experiments presented are governed by a SEED defined in the
configuration. This ensures that the entire ensemble of universes can be
reproduced exactly, which is critical for scientific validation,
debugging, and peer review.

\textbf{Author's Note.} I am an independent researcher. This manuscript
is an exploratory, computational proposal rather than a final theory.
Any inaccuracies are unintentional; the goal is to present a falsifiable
mechanism that invites replication, critique, and refinement by the
broader community.

\subsubsection{TQE E+I Universe Analysis (Run ID:
20250919\_035838)}\label{tqe-ei-universe-analysis-run-id-20250919_035838}

\textbf{Global stability, entropy, and law lock-in metrics for Energy +
Information universes}

This document summarizes the key findings from the TQE E+I simulation
run \texttt{20250919\_035838}. The analysis explores the conditions
required for universe stability and the emergence of physical laws based
on the interplay of Energy (E) and Information (I).

\begin{center}\rule{0.5\linewidth}{0.5pt}\end{center}

\subsection{Mathematical Framework of the
Simulation}\label{mathematical-framework-of-the-simulation}

The TQE framework is built upon a quantitative model designed to
simulate the emergence of stable physical laws from a pre-law quantum
state. The core of the simulation is described by a set of mathematical
equations and operational definitions that govern how Energy (E) and
Information (I) interact to determine a universe's fate.

\subsubsection{1. The Core Modulation
Equation}\label{the-core-modulation-equation}

At the heart of the model is the modulation of a baseline quantum
probability distribution, \(P(ψ)\) , which represents the superposition
of all potential universal states. This distribution is biased by a
fine-tuning function, \(f(E,I)\) , which incorporates the influence of
both vacuum energy fluctuations and informational orientation. The
modulated, post-interaction probability distribution, \(P′(ψ)\) , is
given by:

\[
P′(ψ)=P(ψ)⋅f(E,I)
\]

This equation establishes that the final state of the universe is not a
result of pure chance, but is actively selected based on the interplay
between its energetic and informational content.

\subsubsection{2. The Fine-Tuning
Function}\label{the-fine-tuning-function}

The fine-tuning function, \(f(E,I)\) , combines two distinct physical
hypotheses into a single mathematical form. It consists of an energetic
``Goldilocks'' filter and a linear Information bias term:

\[
f(E,I) = \exp\left(-\frac{(E-E_c)^2}{2\sigma^2}\right) \cdot (1+\alpha I)
\]

The two components of this function are:

2.1 \textbf{The Energetic Goldilocks Zone:} The Gaussian term,
\(\exp\left(-\frac{(E-E_c)^2}{2\sigma^2}\right)\) , ensures that
stability is most probable for universes with an initial energy \(E\)
close to a critical energy \(E_c\). The stability width \(\sigma\)
controls how sensitive the system is to deviations from \(E_c\). In the
simulations analyzed, these were set to \(E_c = 4.0\) and
\(\sigma = 4.0\) .

2.2 \textbf{The Information Bias:} The linear term, \((1 + \alpha I)\) ,
models the hypothesis that Information provides a direct bias towards
ordered outcomes. The orientation bias strength \(\alpha\) (
\(\alpha = 0.8\) in this run) quantifies the strength of this effect.
When \(I > 0\) , the probability of collapse into a
complexity-permitting state is enhanced.

\subsubsection{3. The Information Parameter
(I)}\label{the-information-parameter-i}

The Information parameter \(I\) is defined information-theoretically as
a normalized measure of asymmetry between the probability distributions
of the system at two successive time steps, \(P_t\) and \(P_{t+1}\) .
This is calculated using the Kullback-Leibler (KL) divergence,
\(D_{KL}\), which quantifies the information lost when one distribution
is used to approximate the other. The formula is normalized to ensure
\(0 \le I \le 1\) :

\[
I = \frac{D_{KL}(P_t \parallel P_{t+1})}{1 + D_{KL}(P_t \parallel P_{t+1})}
\]

In this context, a higher value of \(I\) represents a stronger
directional bias in the evolution of the quantum state. The simulation
also explores a composite definition where the KL-derived value is
combined with the Shannon Entropy (H) of the state, often via product
fusion ( \(I = I_{kl} \times I_{shannon}\) ), to create a parameter that
captures both asymmetry and intrinsic complexity.

\subsubsection{4. The Lock-in Criterion}\label{the-lock-in-criterion}

The final, immutable state of ``Law Lock-in'' is not an assumption but
an emergent state identified by a precise operational criterion. A
universe is considered to have achieved Law Lock-in when the relative
variation of its key parameters ( \(\Delta P/P\) ) falls below a
specific threshold for a sustained number of epochs. Based on the
simulation configuration, this is defined as:

\[
\frac{\Delta P}{P} < 0.005\  for\ at\ least\ 6\ consecutive\ epochs.
\]

This criterion \texttt{(REL\_EPS\_LOCKIN\ =\ 0.005},
\texttt{CALM\_STEPS\_LOCKIN\ =\ 6)} provides an objective and
reproducible method for distinguishing universes that successfully
finalize their physical laws from those that remain stable but mutable,
or those that descend into chaos.

\begin{center}\rule{0.5\linewidth}{0.5pt}\end{center}

\subsubsection{Figure 1: Distribution of Simulated Universe Fates in the
E+I
Cohort}\label{figure-1-distribution-of-simulated-universe-fates-in-the-ei-cohort}

This bar chart illustrates the final distribution of outcomes for the
\textbf{10,000 simulated universes} in the E+I cohort, where both Energy
(E) and Information (I) are active parameters. The universes are
classified into three distinct categories based on their long-term
behavior: achieving ``Lock-in,'' achieving stability without lock-in, or
remaining unstable.

\textbf{Analysis:}

The results provide a foundational overview of the ultimate fates of
universes governed by the interplay of Energy and Information.

\begin{enumerate}
\def\labelenumi{\arabic{enumi}.}
\item
  \textbf{Prevalence of Instability}: The most common outcome was
  instability, with \textbf{4,907 universes (49.1\%)} failing to reach a
  stable state. This suggests that the conditions for developing a
  consistent, ordered cosmos are not met in approximately half of the
  cases, even with the guiding influence of Information.
\item
  \textbf{Overall Stability Rate}: A slight majority of universes,
  \textbf{5,093 in total (50.9\%)}, achieved some form of stability.
  This cohort is further divided into two sub-categories, indicating
  different degrees of cosmic resolution.
\item
  \textbf{Hierarchy of Stability}: Within the stable group,
  \textbf{2,884 universes (28.8\%)} reached a ``Stable (no lock-in)''
  state, where their fundamental parameters ceased to fluctuate
  significantly. A smaller but substantial subset of \textbf{2,209
  universes (22.1\%)} achieved the stronger condition of ``Lock-in,''
  where the physical laws themselves became permanently fixed.
\end{enumerate}

\begin{center}\rule{0.5\linewidth}{0.5pt}\end{center}

\subsubsection{Figure 2: The Goldilocks Zone for Universe
Stability}\label{figure-2-the-goldilocks-zone-for-universe-stability}

This plot reveals the relationship between the initial Complexity
parameter \texttt{(X\ =\ E·I)} of a simulated universe and its resulting
probability of achieving a stable outcome. The analysis is based on the
\texttt{10,000} universes from the E+I cohort. The blue points represent
the mean stability rate for universes grouped into discrete bins by
their X-value, while the red spline fit illustrates the overall trend.

\textbf{Analysis:}

The spline curve provides compelling quantitative evidence for a
precisely defined ``Goldilocks Zone''---a narrow range of initial
complexity that is highly conducive to forming a stable universe.

\begin{enumerate}
\def\labelenumi{\arabic{enumi}.}
\item
  \textbf{Optimal Complexity Range}: The analysis identifies a
  well-defined optimal zone for stability, indicated by the vertical
  dashed lines. Universes with an initial complexity value between
  \texttt{X\ =\ 23.09} and \texttt{X\ =\ 27.21} have the highest
  likelihood of evolving into a stable state.
\item
  \textbf{Peak Stability}: The curve reaches its global maximum at an
  initial complexity of \texttt{X\ =\ 25.56}. At this specific value,
  the probability of a universe achieving stability is maximal,
  approaching approximately 95\%.
\item
  \textbf{High Sensitivity}: The probability of stability drops sharply
  outside of the optimal zone. Universes with too little complexity
  (\texttt{X\ \textless{}\ 20}) or too much
  (\texttt{X\ \textgreater{}\ 30}) are significantly less likely to
  become stable. This highlights the fine-tuned nature of this
  parameter. A smaller, secondary peak is observed near
  \texttt{X\ ≈\ 32}, but it represents a much lower probability of
  success.
\end{enumerate}

\textbf{Key Insight}: This figure powerfully demonstrates that a
universe's fate is critically sensitive to the initial balance between
its Energy and Information content, as quantified by the Complexity
parameter \texttt{X}. The existence of a sharp, narrow peak underscores
the core tenet of fine-tuning; it is not merely the presence of E and I,
but their specific multiplicative relationship that is paramount. Both
an excess and a deficit of this combined ``complexity factor'' are
overwhelmingly detrimental to the formation of a stable cosmos within
the TQE framework.

\subsubsection{Synthesis: Why the ``Best'' Universes Sit Below the
Stability
Peak}\label{synthesis-why-the-best-universes-sit-below-the-stability-peak}

Figure 2 identifies the peak \textbf{stability probability} near (X
\approx 25.6). However, the ``best'' universes (Figures 13--15)
originate at lower (X) (≈12--15). This is not a contradiction but
evidence of a \textbf{two-factor selection}:

\begin{itemize}
\tightlist
\item
  \textbf{Gate (Stability):} (E\cdot I) must place a universe inside the
  Goldilocks window to avoid chaos.
\item
  \textbf{Trigger (Finality):} A sufficiently large asymmetry
  (\textbar E-I\textbar) sharply increases the chance of \textbf{rapid
  law lock-in}.
\end{itemize}

The most successful outcomes are therefore \textbf{sub-peak, not
super-peak}: they are \emph{stable enough} to pass the gate, yet
\emph{asymmetric enough} to trigger early finality. This explains why
the top universes concentrate just below the stability maximum while
still outperforming in early lock-in and global coherence.

\begin{center}\rule{0.5\linewidth}{0.5pt}\end{center}

\subsubsection{Figure 3: Stability Outcomes in the Energy-Information
Parameter
Space}\label{figure-3-stability-outcomes-in-the-energy-information-parameter-space}

This scatter plot visualizes the initial parameter space for all 10,000
universes simulated in the E+I cohort. Each point represents a single
universe, positioned according to its initial \textbf{Energy (E)} on the
x-axis and \textbf{Information (I)} on the y-axis. The color of each
point indicates its final evolutionary outcome, as shown by the color
bar: \textbf{red for stable universes (1) and blue for unstable
universes (0)}.

Analysis:

The plot reveals the complex, non-linear relationship between the
initial \texttt{E} and \texttt{I} parameters and the likelihood of a
universe achieving stability.

\begin{enumerate}
\def\labelenumi{\arabic{enumi}.}
\item
  \textbf{Parameter Distributions}: The initial conditions show a
  distinct distribution pattern. The Energy (E) parameter is heavily
  skewed toward lower values (most universes have
  \texttt{E\ \textless{}\ 50}), with a long tail of high-energy
  outliers, which is characteristic of a log-normal distribution. The
  Information (I) parameter is more evenly distributed across its range.
\item
  \textbf{Low-Energy Regime}: In the region of low energy (approximately
  \texttt{E\ \textless{}\ 50}), where the vast majority of universes are
  instantiated, there is a dense intermixing of both stable (red) and
  unstable (blue) outcomes. This indicates that for low-E universes, the
  Energy value alone is a poor predictor of stability; the outcome is
  critically dependent on the corresponding Information value.
\item
  \textbf{High-Energy Trend}: A clear trend emerges at higher energy
  levels (roughly \texttt{E\ \textgreater{}\ 75}). Unstable (blue)
  universes become exceedingly rare in this regime. Although fewer
  universes are generated with such high initial energy, those that are
  have a very high probability of evolving into a stable state.
\item
  \textbf{Implicit Goldilocks Correlation}: While this plot does not
  explicitly graph the Complexity \texttt{X\ =\ E·I}, the concentration
  of stable red points forms a discernible pattern. This pattern is
  consistent with the hyperbolic curve (\texttt{I\ =\ constant\ /\ E})
  that would define the optimal ``Goldilocks Zone'' identified in Figure
  2. The most successful universes are not found at the extremes of
  either E or I, but within a specific combinatoric region that balances
  the two.
\end{enumerate}

\textbf{Key Insight}: This visualization powerfully illustrates that
cosmic stability in the TQE model is not a simple monotonic function of
either Energy or Information. While very high energy appears to be a
sufficient condition for stability, it is a rare initial state. For the
majority of universes born in the more common low-energy regime, a
delicate balance with the Information parameter is required to achieve
an ordered outcome. This plot decomposes the abstract ``Goldilocks
Zone'' of Complexity into its constituent parts, reinforcing the central
thesis that it is the \textbf{interplay and fine-tuned balance between E
and I}, not just their individual magnitudes, that fundamentally
determines a universe's potential for a stable existence.

\begin{center}\rule{0.5\linewidth}{0.5pt}\end{center}

\subsubsection{Figure 4: Statistical Evolution During the Quantum
Fluctuation
Phase}\label{figure-4-statistical-evolution-during-the-quantum-fluctuation-phase}

This plot displays the time evolution of the expectation value
(\texttt{⟨A⟩}, mean) and variance (\texttt{Var(A)}) of a fundamental
observable `A' during the initial quantum fluctuation stage of the
simulation. These curves represent the averaged behavior across the
entire 10,000-universe E+I cohort, providing insight into the primordial
state from which each universe emerges.

\textbf{Analysis:}

The graph demonstrates a swift and decisive process of initial
state-setting common to all simulated universes.

\begin{enumerate}
\def\labelenumi{\arabic{enumi}.}
\item
  \textbf{Rapid Convergence}: The most prominent feature is the rapid
  stabilization of both statistical moments. Within the first unit of
  simulation time (\texttt{t\ \textless{}\ 1.0}), both the expectation
  value and the variance converge to stable equilibrium values, after
  which they remain constant.
\item
  \textbf{Zero-Mean Expectation Value}: The expectation value
  \texttt{⟨A⟩} (dashed blue line), following a brief initial dip,
  quickly converges to and holds a value of \textbf{zero}. This
  indicates that the quantum fluctuations are, on average, unbiased,
  producing no net positive or negative value for the observable A. This
  is consistent with an origin from pure, directionless quantum noise.
\item
  \textbf{Normalized Variance}: The variance \texttt{Var(A)} (solid
  orange line) begins at a high value (\texttt{≈\ 0.95}) and rapidly
  converges to a stable value of \textbf{one}. A system that resolves to
  a state with a mean of \texttt{0} and a variance of \texttt{1} is
  characteristic of a standardized random variable, suggesting the
  fluctuation phase establishes a normalized and statistically
  well-behaved foundation.
\end{enumerate}

\textbf{Key Insight}: This figure illustrates the critical function of
the quantum fluctuation phase: to serve as a robust initialization
mechanism. It ensures that every universe, regardless of its specific E
and I parameters, begins its evolutionary journey from a consistent and
statistically standardized state, analogous to a normalized vacuum. The
rapid convergence implies that this state-setting is a foundational and
highly efficient process, creating a stable ``blank slate'' of unbiased
randomness from which the more complex, parameter-dependent evolution
toward stability or instability can subsequently unfold.

\begin{center}\rule{0.5\linewidth}{0.5pt}\end{center}

\subsubsection{Figure 5: Evolution of Entropy and Purity during the
Quantum Superposition
Phase}\label{figure-5-evolution-of-entropy-and-purity-during-the-quantum-superposition-phase}

This graph plots the average behavior of two fundamental
quantum-informational metrics---\textbf{Entropy} (blue dashed line) and
\textbf{Purity} (orange dashed line)---over time during the quantum
superposition phase. This stage models the universe as a complex
superposition of states, preceding the collapse into a single, classical
reality.

\textbf{Analysis:}

The plot illustrates a rapid and dramatic transformation from a state of
perfect order to one of maximal complexity and uncertainty.

\begin{enumerate}
\def\labelenumi{\arabic{enumi}.}
\item
  \textbf{Inverse Correlation}: A stark inverse relationship exists
  between Entropy and Purity. At \texttt{t=0}, the system begins in a
  state of maximum Purity (1.0) and minimum Entropy (near 0). As the
  simulation progresses, Purity rapidly declines while Entropy
  reciprocally increases.
\item
  \textbf{Entropy Saturation}: The Entropy of the system skyrockets from
  nearly zero to a maximum value of approximately \textbf{1.0} within
  the first 1.5 units of time. After this point, it fluctuates around
  this maximal value, indicating the system has reached a state of
  maximum statistical mixedness or uncertainty.
\item
  \textbf{Purity Collapse}: Correspondingly, the Purity of the state
  begins at 1.0 (a perfectly pure state) and collapses to a fluctuating
  low value of around 0.25. This demonstrates that the initial,
  well-defined state has evolved into a highly mixed state, representing
  a complex superposition of numerous potential realities.
\end{enumerate}

\textbf{Key Insight}: This figure models the creation of the quantum
``state-space'' for the nascent universes. The simulation initializes
each cosmos in a simple, pure state of low information content (low
entropy). The superposition process then rapidly evolves this into a
maximally mixed state, representing a rich sea of possibilities where
all potential outcomes coexist. This high-entropy, low-purity
equilibrium is a critical step, establishing the quantum substrate from
which a single, classical universe is selected during the subsequent
``collapse'' phase of its evolution. The speed of this transition
suggests it is a foundational and universal feature of cosmic genesis in
the TQE model.

\begin{center}\rule{0.5\linewidth}{0.5pt}\end{center}

\subsubsection{Figure 6: The Collapse Event and Initial Parameter
Fixation of
X}\label{figure-6-the-collapse-event-and-initial-parameter-fixation-of-x}

This time-series plot illustrates the pivotal ``collapse'' event,
centered at \texttt{t=0}. It shows the universe's fundamental Complexity
parameter (\texttt{X\ =\ E·I}) transitioning from a volatile,
indeterminate quantum phase to a stable, classical phase with a single,
definite value. The gray line tracks the value of \texttt{X}, averaged
for a cohort of universes undergoing this transition.

\textbf{Analysis:}

The graph provides a clear depiction of the universe's transition from a
multi-potential quantum state to a singular classical reality.

\begin{enumerate}
\def\labelenumi{\arabic{enumi}.}
\item
  \textbf{Pre-Collapse Superposition (\texttt{t\ \textless{}\ 0})}:
  Before the event at \texttt{t=0}, the Complexity \texttt{X} lacks a
  definite value, exhibiting large and erratic fluctuations. This
  represents the quantum superposition state where a wide range of
  potential \texttt{X} values coexist.
\item
  \textbf{The Collapse Instant (\texttt{t\ =\ 0})}: Marked by the
  vertical red line, this is the moment of decoherence. The
  superposition of states resolves, and a single, specific value for the
  Complexity parameter is selected.
\item
  \textbf{Post-Collapse Parameter Fixation
  (\texttt{t\ \textgreater{}\ 0})}: Immediately following the collapse,
  the system's behavior changes fundamentally. The large-scale quantum
  fluctuations vanish and are replaced by low-amplitude, classical-like
  oscillations around a now-stable mean.
\item
  \textbf{Resulting \texttt{X} Parameter}: The horizontal red dashed
  line indicates the specific value to which X has collapsed. For this
  representative case, the universe proceeds its evolution governed by
  the fixed fundamental parameter of \texttt{X\ =\ 3.70}.
\end{enumerate}

\textbf{Key Insight}: This figure models the crucial step of parameter
fixation, the TQE model's analog for a quantum measurement or wave
function collapse. This event is the necessary precondition for, but
distinct from, the later evolutionary state of ``Law Lock-in.'' At
\texttt{t=0}, the universe's fundamental nature, defined by its
\texttt{X} value, becomes fixed. This newly established constant then
governs the universe's entire subsequent evolution during the expansion
phase. Whether that evolution will ultimately lead to the final,
immutable state of ``Law Lock-in'' is determined by the specific value
of \texttt{X} that was selected in this primordial collapse event.

\begin{center}\rule{0.5\linewidth}{0.5pt}\end{center}

\subsubsection{Figure 7: Expansion Dynamics and the Average Epoch of Law
Lock-in}\label{figure-7-expansion-dynamics-and-the-average-epoch-of-law-lock-in}

This plot illustrates the post-collapse dynamics for the E+I universe
cohort, focusing on the expansion phase from \texttt{epoch\ 0} to
\texttt{800}. The graph tracks the evolution of the universe's expansion
(\texttt{Amplitude\ A}) and a secondary \texttt{Orientation\ I}
parameter. Crucially, it marks the average epoch at which a subset of
these universes achieves the final ``Law Lock-in'' state.

\textbf{Analysis:}

The chart showcases the primary long-term evolutionary behavior of the
simulated universes after their fundamental \texttt{X} parameter has
been fixed.

\begin{enumerate}
\def\labelenumi{\arabic{enumi}.}
\item
  \textbf{Sustained Cosmic Expansion}: \texttt{The\ Amplitude\ A} (blue
  line) demonstrates a clear and persistent growth trend, representing
  the ongoing expansion of the universe's scale. The expansion appears
  roughly linear with small stochastic fluctuations, indicating a steady
  increase in size over time.
\item
  \textbf{Parameter Decoupling}: The \texttt{Orientation\ I} parameter
  (orange line) remains stable and close to zero throughout the entire
  epoch. This suggests that this particular parameter is decoupled from
  the expansion dynamics, having either been fixed at \texttt{t=0} or
  being a conserved quantity within the model.
\item
  \textbf{Late-Stage Law Lock-in}: The vertical red dashed line
  identifies the average epoch for ``Law Lock-in'' at approximately
  \texttt{t\ =\ 747}. Based on the provided context, this is a mean
  value calculated only from the subset of universes (22.1\% of the
  total) that successfully reached this ultimate state of stability.
  This confirms that Law Lock-in is not an initial condition but an
  emergent state achieved late in a universe's evolution.
\item
  \textbf{Independence of Expansion and Finality}: A critical
  observation is that the universe's expansion (\texttt{Amplitude\ A})
  continues unabated through and beyond the point of Law Lock-in. The
  freezing of the universe's fundamental physical rules does not halt or
  alter the metric expansion of spacetime.
\end{enumerate}

\textbf{Key Insight}: This figure provides a crucial distinction between
a universe's dynamic evolution and the finalization of its laws. The
``Law Lock-in'' event is shown to be a late-stage achievement for a
fraction of universes, not a universal starting point. The most
significant insight is the \textbf{decoupling of physical law finality
from spatial expansion}. In the TQE model, a universe can solidify its
fundamental constants and rules while its spatial fabric continues to
expand. This suggests that a universe's core identity (\texttt{laws})
and its dynamic behavior (\texttt{expansion}) are distinct and can
operate on different evolutionary timescales.

\begin{center}\rule{0.5\linewidth}{0.5pt}\end{center}

\subsubsection{Figure 8: Entropy Evolution and Regional Convergence in a
High-Performing Universe (ID
7804)}\label{figure-8-entropy-evolution-and-regional-convergence-in-a-high-performing-universe-id-7804}

This plot provides a microscopic view of the entropy dynamics within a
single, top-ranked ``best'' universe (ID 7804) from the E+I cohort. It
contrasts the evolution of the system's \textbf{Global Entropy} (thick
black line) with the entropies of eight distinct sub-regions within it
(thin colored lines). The vertical dashed line indicates the precise
time step where this universe achieved ``Law Lock-in.''

\textbf{Analysis}:

The graph reveals two distinct modes of entropy evolution---global and
regional---and highlights the transformative nature of the Law Lock-in
event.

\begin{enumerate}
\def\labelenumi{\arabic{enumi}.}
\item
  \textbf{Monotonic Global Entropy Growth}: The global entropy of the
  universe follows a smooth, predictable trajectory. It begins at a high
  value (\texttt{≈5.6}) and asymptotically approaches a maximum state of
  around 6.0. This represents the orderly and continuous increase in the
  overall information content or complexity of the universe as a whole.
\item
  \textbf{Primordial Regional Chaos}: In the early epochs
  (\texttt{t\ \textless{}\ 305}), the entropies of the individual
  regions are highly volatile and divergent. They fluctuate erratically
  and independently, indicating a primordial state where different
  domains of the universe have not yet settled into a coherent, unified
  physical regime.
\item
  \textbf{Law Lock-in as a Coherence Event}: The ``Law Lock-in'' at
  \texttt{epoch\ ≈\ 305} marks a dramatic phase transition. At this
  moment, the chaotic behavior of the regional entropies abruptly
  ceases. They rapidly converge to a single, stable value of
  approximately 5.1, demonstrating the imposition of a uniform set of
  physical laws across the entire cosmic space.
\item
  \textbf{Long-Term Stability}: Following the lock-in event, the
  now-unified regional entropies remain remarkably stable and coherent
  for the rest of the simulation's duration, staying well above the
  minimum stability threshold (red dashed line at 3.5).
\end{enumerate}

\textbf{Key Insight}: This figure provides a powerful visualization of
``Law Lock-in'' not merely as a statistical outcome, but as a
\textbf{dynamic, coherence-imposing event}. The mechanism transforms a
universe of disconnected, chaotic regions into a unified,
self-consistent cosmos governed by a single set of laws. The divergence
between the ever-increasing global entropy and the stabilized regional
entropies is particularly significant. It suggests a model where the
\textbf{lock-in of laws creates a stable, predictable foundation at the
local level}---a necessary condition for the formation of
structure---while still permitting the \textbf{overall complexity of the
universe to grow}.

\begin{center}\rule{0.5\linewidth}{0.5pt}\end{center}

\subsubsection{Figure 9: Entropy Dynamics of the Second-Ranked
High-Performing Universe (ID
1421)}\label{figure-9-entropy-dynamics-of-the-second-ranked-high-performing-universe-id-1421}

This plot presents the entropy evolution for the second-ranked ``best''
universe (ID 1421), displaying the same metrics as the previous figure:
the overall \textbf{Global Entropy} (black line) and the individual
entropies of its eight constituent sub-regions (colored lines).

\textbf{Analysis}:

The evolutionary trajectory of this universe is remarkably similar to
that of the top-ranked universe, reinforcing the patterns observed
previously.

\begin{enumerate}
\def\labelenumi{\arabic{enumi}.}
\item
  \textbf{Consistent Dynamics}: This universe exhibits an evolutionary
  profile nearly identical to the one shown in Figure 8. The global
  entropy follows a smooth curve of monotonic growth, while the regional
  entropies begin in a state of high volatility and divergence.
\item
  \textbf{Early and Decisive Lock-in}: The pivotal ``Law Lock-in'' event
  occurs at \texttt{epoch\ ≈\ 306}, almost precisely mirroring the
  timing of the first-ranked universe. This event again functions as a
  sharp phase transition, immediately quelling the regional chaos and
  forcing all sub-regions into a coherent state.
\item
  \textbf{Stable Convergence}: Post lock-in, all eight regional
  entropies converge to a stable, shared value of approximately 5.1,
  which is maintained for the remainder of the simulation and remains
  well above the 3.5 stability threshold.
\end{enumerate}

\textbf{Key Insight}: This figure strengthens the conclusions drawn from
the previous analysis. The striking similarity in the evolutionary paths
of the two top-ranked universes suggests that \textbf{an early and rapid
``Law Lock-in'' event is a defining characteristic of the most
successful and robustly stable outcomes} generated by the TQE model.
This repeated pattern indicates that the model favors a mechanism where
a chaotic, multi-regional primordial state is swiftly unified into a
homogeneous and stable cosmos. This rapid coherence appears to be a key
feature for creating a viable universe within this framework.

\begin{center}\rule{0.5\linewidth}{0.5pt}\end{center}

\subsubsection{Figure 10: Entropy Dynamics of the Third-Ranked
High-Performing Universe (ID
3806)}\label{figure-10-entropy-dynamics-of-the-third-ranked-high-performing-universe-id-3806}

This chart concludes the series of ``best-universe'' analyses by
plotting the entropy evolution for the third-ranked successful universe
(ID 3806). As with the previous examples, it contrasts the
\textbf{Global Entropy} (black line) with the entropies of eight
internal sub-regions (colored lines) and highlights the moment of ``Law
Lock-in.''

\textbf{Analysis}:

This universe's evolutionary path provides further confirmation of the
archetypal behavior observed in the top-ranked outcomes.

\begin{enumerate}
\def\labelenumi{\arabic{enumi}.}
\item
  \textbf{Archetypal Evolution}: The overall dynamics are consistent
  with the previous two figures. Global entropy increases smoothly
  towards a maximum, while the regional entropies display an initial
  phase of high-amplitude, uncorrelated fluctuations.
\item
  \textbf{Slightly Delayed Lock-in}: The ``Law Lock-in'' event for this
  universe occurs at \texttt{epoch\ ≈\ 311}. This is still very early in
  the simulation's 3000-epoch timeline but is marginally later than the
  lock-in times of the first (≈305) and second (≈306) ranked universes.
\item
  \textbf{Convergence to Stability}: Following the lock-in event, the
  regional entropies once again undergo a rapid transition, converging
  to a common, stable value around \textasciitilde5.1 that is well above
  the stability threshold.
\end{enumerate}

\textbf{Key Insight}: This third example solidifies the conclusion that
\textbf{early and decisive law lock-in is the defining feature of the
most successful universes} in the TQE model. The consistent pattern
across all three top-ranked universes---a rapid phase transition from
regional quantum chaos to global classical order occurring around epoch
300---suggests this is a robust and primary pathway to a stable cosmos.

The subtle trend in the lock-in times (rank 1 at 305, rank 2 at 306,
rank 3 at 311) hints at a potential selection criterion: \textbf{the
``best'' universes may be those that achieve coherence and finalize
their laws the fastest}, thereby establishing a stable foundation for
structure formation as early as possible.

\begin{center}\rule{0.5\linewidth}{0.5pt}\end{center}

\subsubsection{Figure 11: Lock-in Probability as a Function of the
Energy-Information Gap \textbar E −
I\textbar{}}\label{figure-11-lock-in-probability-as-a-function-of-the-energy-information-gap-e-i}

This plot investigates a key fine-tuning relationship within the TQE
model, showing the probability of a universe achieving ``Law Lock-in''
as a function of the initial \textbf{Energy-Information gap} (defined as
\texttt{\textbar{}E\ −\ I\textbar{}}). The data points represent the
lock-in probability calculated for discrete bins of
\texttt{\textbar{}E\ −\ I\textbar{}} values across the E+I cohort, with
error bars indicating the statistical uncertainty in each bin.

\textbf{Analysis}:

The results demonstrate a clear and strong relationship between the
initial parameter imbalance and the ultimate fate of a universe.

\begin{enumerate}
\def\labelenumi{\arabic{enumi}.}
\item
  \textbf{Positive Correlation}: The most striking feature is the
  strong, positive correlation between the
  \texttt{\textbar{}E\ −\ I\textbar{}} gap and \texttt{P(lock-in)}. As
  the absolute difference between the initial Energy and Information
  parameters increases, so does the probability of the universe
  achieving a final, immutable ``Lock-in'' state.
\item
  \textbf{Suppression of Lock-in at E≈I}: When the values of Energy and
  Information are nearly balanced (\texttt{\textbar{}E\ −\ I\textbar{}}
  approaches zero), the probability of achieving lock-in is minimal, at
  only a few percent. This suggests that a state of symmetry or balance
  between these two fundamental parameters is not conducive to the
  finalization of physical laws.
\item
  \textbf{Region of Rapid Growth}: The lock-in probability rises most
  steeply in the range of
  \texttt{0\ \textless{}\ \textbar{}E\ −\ I\textbar{}\ \textless{}\ 30},
  increasing from near-zero to approximately 40\%. This indicates a high
  sensitivity to initial imbalances in the low-gap regime.
\item
  \textbf{Trend at High Imbalance}: For universes with a large
  \texttt{\textbar{}E\ −\ I\textbar{}} gap, the probability of lock-in
  continues to rise, exceeding 50\% for the largest imbalances observed
  (\texttt{\textbar{}E\ −\ I\textbar{}\ \textgreater{}\ 120}). This
  implies that a state of clear dominance by one parameter (typically
  Energy, given its log-normal distribution) is highly favorable for
  producing a universe with immutable laws.
\end{enumerate}

\textbf{Key Insight}: This analysis reveals a crucial and non-trivial
aspect of the model's fine-tuning dynamics. While the product
\texttt{E·I} (Complexity) must fall within a narrow Goldilocks Zone for
a universe to be stable, this plot shows that the difference
\texttt{\textbar{}E\ −\ I\textbar{}} is a primary driver for achieving
the ultimate finality of \textbf{Law Lock-in}.

The model suggests that universes born ``in balance'' (\texttt{E\ ≈\ I})
may successfully stabilize but are unlikely to ever have their physical
laws fully ``freeze.'' Instead, it is the universes with a significant
\textbf{Energy-Information asymmetry} that are preferentially selected
for a fate with immutable, locked-in laws. This implies that finality is
not born from equilibrium, but from a decisive imbalance between the
universe's core energetic and informational components.

\subsubsection{Figure 12: Comparative Lock-in Probability for High
vs.~Low Energy-Information
Gap}\label{figure-12-comparative-lock-in-probability-for-high-vs.-low-energy-information-gap}

This bar chart provides a direct comparison of the ``Law Lock-in''
probability between two distinct populations of universes. The data is
bifurcated based on an adaptively determined threshold for the
Energy-Information gap (\texttt{\textbar{}E\ −\ I\textbar{}}). The left
bar represents universes with a significant E-I imbalance
(\texttt{\textbar{}E−I\textbar{}\ \textgreater{}\ 5.86}), while the
right bar represents those where E and I are approximately balanced
(\texttt{\textbar{}E−I\textbar{}\ ≤\ 5.86}).

\textbf{Analysis}:

The plot offers a clear, quantitative confirmation of the role of
parameter asymmetry in achieving a universe's final, immutable state.

\begin{enumerate}
\def\labelenumi{\arabic{enumi}.}
\item
  \textbf{Asymmetry Drives Lock-in}: The data shows unequivocally that
  universes with a significant imbalance between their Energy and
  Information values are far more likely to achieve Law Lock-in. This
  group (\texttt{\textbar{}E−I\textbar{}\ \textgreater{}\ 5.86})
  exhibits a high lock-in probability of approximately 26\%.
\item
  \textbf{Balance Suppresses Lock-in}: Conversely, universes where the E
  and I parameters are numerically close
  (\texttt{\textbar{}E−I\textbar{}\ ≤\ 5.86}) show a drastically reduced
  probability of finalizing their physical laws. The likelihood for this
  cohort is only about 6\%.
\item
  \textbf{Magnitude of the Effect}: The comparison highlights a powerful
  relationship. A universe with a large E-I imbalance is more than four
  times as likely to reach the Law Lock-in state than a universe where
  these parameters are balanced. The small error bars indicate high
  confidence in this result.
\end{enumerate}

\textbf{Key Insight}: This figure distills the finding from the previous
analysis into a stark and unambiguous conclusion:
\textbf{Energy-Information asymmetry is a primary driver of ``Law
Lock-in''}. While the product (\texttt{E·I}) governs the potential for
initial stability, the difference (\texttt{\textbar{}E-I\textbar{}})
appears to be a key selection mechanism for ultimate finality. The model
suggests that for a universe's physical laws to become permanently
fixed, a state of equilibrium between its foundational components is
insufficient; a decisive imbalance is required to break the symmetry and
propel the system into a single, unchanging state.

\begin{center}\rule{0.5\linewidth}{0.5pt}\end{center}

\subsubsection{Figure 13-15: Simulated Cosmic Microwave Background (CMB)
Anisotropies for the Top Three ``Best''
Universes}\label{figure-13-15-simulated-cosmic-microwave-background-cmb-anisotropies-for-the-top-three-best-universes}

This figure presents the simulated Cosmic Microwave Background (CMB)
temperature anisotropy maps for the three top-ranked universes
identified by the simulation: \textbf{UID 7804 (Rank 1), UID 1421 (Rank
2), and UID 3806 (Rank 3)}. These all-sky maps, presented in a Mollweide
projection, serve as the model's most direct diagnostic output for
comparison with observational cosmology. The temperature fluctuations
(anisotropies) are shown in units of micro-Kelvin (µK).

\subsubsection{Figure 13: Best CMB, Rank 1 (UID
7804)}\label{figure-13-best-cmb-rank-1-uid-7804}

\subsubsection{Figure 14: Best CMB, Rank 2 (UID
1421)}\label{figure-14-best-cmb-rank-2-uid-1421}

\subsubsection{Figure 15: Best CMB, Rank 3 (UID
3806)}\label{figure-15-best-cmb-rank-3-uid-3806}

\subsection{Analysis: A Two-Factor Selection for Optimal
Universes}\label{analysis-a-two-factor-selection-for-optimal-universes}

A comparative analysis of these three high-performing universes reveals
consistent patterns and provides deeper insight into the criteria for a
``successful'' cosmogenesis in the TQE model. The key to understanding
their origin lies not in a single parameter, but in a two-factor
selection mechanism that governs both stability and finality.

At a visual level, all three CMB maps display a statistically isotropic
and Gaussian-like distribution of hot and cold spots, which is
qualitatively consistent with the observed CMB from our own universe.
This indicates that the model is capable of producing cosmologically
plausible outputs.

The parameters of these top-ranked universes reveal the dual selection
criteria at play. Their respective Complexity values (X = E·I) of
approximately 12.0, 13.6, and 15.0 lie significantly lower than the peak
probability region of the ``Goldilocks Zone'' for stability, which
peaked at X ≈ 25.6. However, their Energy--Information gaps (\textbar E
− I\textbar) are all large (\textgreater26). This is not a paradox, but
a direct consequence of the model's dual selection pressures:

\begin{itemize}
\tightlist
\item
  \textbf{The Stability Gate (E·I):} A universe must first possess a
  viable X value to pass through the ``gate'' and have a chance at
  stability.\\
\item
  \textbf{The Lock-in Trigger (\textbar E − I\textbar):} From the pool
  of stable candidates, those with a significant Energy--Information
  asymmetry are preferentially selected to undergo a rapid and decisive
  ``Law Lock-in''.
\end{itemize}

This mechanism is further supported by the lock-in timing, which acts as
a primary success metric. The rank of the universe correlates directly
with the speed of its law finalization: Rank 1 (≈305 epochs), Rank 2
(≈306 epochs), and Rank 3 (≈311 epochs). This strongly suggests that an
early finalization of physical laws is a primary characteristic of the
most ``successful'' universes in the simulation.

\textbf{Key Insight:} This analysis refines our understanding of what
constitutes an optimal outcome in the TQE framework. The ``best''
universes are not simply those with the highest a priori probability of
stability. Instead, they represent an optimal compromise: their
parameters are ``stable enough'' to survive the initial chaotic phase,
yet ``asymmetric enough'' to trigger the rapid finalization of their
physical laws. This two-factor process explains why the most successful
outcomes are pushed into a specific, sub-peak region of the parameter
space. These CMB maps represent the model's most direct point of contact
with empirical science, and a detailed statistical analysis of their
properties provides the ultimate testing ground for the TQE's
falsifiable predictions.

\begin{center}\rule{0.5\linewidth}{0.5pt}\end{center}

\subsubsection{Figure 16-19: Simulated ``Axis of Evil'' (AoE) Anomalies
in High-Performing
Universes}\label{figure-16-19-simulated-axis-of-evil-aoe-anomalies-in-high-performing-universes}

This figure presents the results of the ``Axis of Evil'' (AoE) analysis,
a key diagnostic test for the TQE model's predictions as outlined in the
manuscript. The analysis searches for anomalous alignments between the
quadrupole and octupole moments in the simulated Cosmic Microwave
Backgrounds (CMBs). The figure includes the CMB maps for the three
universes where this anomaly was detected, alongside a histogram
summarizing their alignment angles.

\subsubsection{Figure 16, 17, 18: Examples of Simulated CMBs with
Measured
Alignments}\label{figure-16-17-18-examples-of-simulated-cmbs-with-measured-alignments}

(These three images show universes selected for AoE analysis, with their
calculated quadrupole-octupole alignment angles noted in their titles.)

\subsubsection{Figure 19, Distribution of the three detected AoE
alignment angles from the E+I
cohort.}\label{figure-19-distribution-of-the-three-detected-aoe-alignment-angles-from-the-ei-cohort.}

\textbf{Analysis}:

The search for large-scale CMB anomalies provides a direct method for
testing the falsifiable predictions of the TQE model. The results from
the 10,000-universe E+I cohort are as follows:

\begin{enumerate}
\def\labelenumi{\arabic{enumi}.}
\item
  \textbf{Rarity of the Anomaly}: The AoE anomaly is a rare event within
  the simulation. It was positively detected in only \textbf{3 out of
  10,000 universes}. Notably, these three universes are the same
  ``best'' universes (UIDs 7804, 1421, 3806) identified previously by
  their rapid and robust stabilization, suggesting a deep connection
  between the mechanism of early Law Lock-in and the generation of these
  large-scale cosmic features.
\item
  \textbf{Alignment Angle Distribution}: The histogram shows the
  specific quadrupole-octupole alignment angles produced by these three
  anomalous universes. The model generated three distinct and widely
  separated outcomes: \textbf{20.1°}, \textbf{54.3°}, and
  \textbf{140.8°}.
\item
  \textbf{Correspondence with Observational Data}: The most significant
  result of this analysis is the alignment angle of ≈20° produced by
  universe UID 1421. This simulated value shows a remarkable
  correspondence with the reference alignment angle of \texttt{≈20°}
  derived from observational data of our own universe's CMB.
\end{enumerate}

\begin{itemize}
\tightlist
\item
  \textbf{Author's Note}: As requested for clarification, the reference
  line in the histogram is incorrectly labeled at ≈10°; the correct
  observational value it is intended to represent is \texttt{≈20°},
  which aligns closely with one of the simulated outcomes.
\end{itemize}

\textbf{Key Insight}: This analysis represents a successful preliminary
test of the TQE model's primary falsifiable prediction. The simulation
is not only capable of generating CMBs with AoE-type anomalies, but in
at least one instance, it has produced an outcome that
\textbf{quantitatively matches a key feature of our observed cosmos.}

The rarity of the anomaly, combined with its appearance exclusively in
the ``best'' universes, suggests that the very dynamics of a rapid and
decisive Law Lock-in may be responsible for imprinting these
large-scale, non-random signatures onto the CMB. While this is not
definitive proof, this ``hit'' provides significant encouragement for
the model's potential validity and warrants a more rigorous statistical
analysis (e.g., power spectrum analysis) of the full ensemble of
simulated CMBs to further test this hypothesis.

\begin{center}\rule{0.5\linewidth}{0.5pt}\end{center}

\subsubsection{Figure 20-24: Analysis of Simulated CMB Cold Spot
Anomalies}\label{figure-20-24-analysis-of-simulated-cmb-cold-spot-anomalies}

This figure presents a comprehensive analysis of the CMB Cold Spot
anomaly as produced by the TQE model in the E+I cohort. The figure
combines three distinct visualizations:

\subsubsection{Figure 20, 21, 22: CMB maps for the three universes where
a statistically significant cold spot was detected (UIDs 7804, 1421,
3806).}\label{figure-20-21-22-cmb-maps-for-the-three-universes-where-a-statistically-significant-cold-spot-was-detected-uids-7804-1421-3806.}

\subsubsection{Figure 23, A histogram comparing the depth (z-score) of
these simulated spots to the observed value of the Planck Cold
Spot.}\label{figure-23-a-histogram-comparing-the-depth-z-score-of-these-simulated-spots-to-the-observed-value-of-the-planck-cold-spot.}

\subsubsection{Figure 24, A heatmap showing the positional distribution
of the detected cold spots on the celestial
sphere.}\label{figure-24-a-heatmap-showing-the-positional-distribution-of-the-detected-cold-spots-on-the-celestial-sphere.}

\textbf{Analysis}:

The model was tested for its ability to reproduce large-scale cold spot
anomalies, another key falsifiable prediction mentioned in the
manuscript.

\begin{enumerate}
\def\labelenumi{\arabic{enumi}.}
\item
  \textbf{Rarity and Correlation with ``Best'' Universes}: Similar to
  the AoE anomaly, the emergence of a significant cold spot is a rare
  phenomenon in the simulation. It was detected in the \textbf{exact
  same 3 out of 10,000 universes} (UIDs 7804, 1421, 3806) that were
  identified as ``best'' outcomes and which also exhibited an AoE. This
  strongly suggests a common physical origin for these anomalies, likely
  linked to the rapid law stabilization process that characterizes these
  specific universes.
\item
  \textbf{Depth Exceeds Observational Data}: The histogram provides the
  most critical result of this analysis. The model successfully
  generates universes with prominent cold spots, with depths
  corresponding to z-scores of approximately \textbf{-68.5, -73.5, and
  -78}. However, all three of these simulated anomalies are
  significantly \textbf{colder and more extreme} than the actual Planck
  Cold Spot, whose value is approximately \textbf{z = -70} (indicated by
  the red reference line).
\item
  \textbf{Random Positional Distribution}: The heatmap shows the
  celestial coordinates of the three detected cold spots. Their
  positions appear to be randomly scattered across the sky, with no
  evident clustering or preferred location. This suggests that while the
  TQE mechanism may allow for the existence of such an anomaly, its
  specific location is a stochastic outcome.
\end{enumerate}

\textbf{Key Insight}: This analysis represents a nuanced and valuable
test of the TQE model. On one hand, the model succeeds in generating
rare, large-scale cold spots, and the fact that these co-occur with the
AoE anomaly in the ``best'' universes is a significant, non-trivial
result pointing to a unified underlying mechanism.

On the other hand, the quantitative discrepancy in the anomaly's
magnitude is a critical finding. The model, in its current
configuration, consistently ``overshoots'' the observed data, producing
cold spots that are even more anomalous than the one in our own cosmos.
This provides a clear and actionable direction for future model
refinement: the parameters or functions governing the collapse dynamics
may need adjustment to temper the magnitude of the resulting anomalies.
This result, while not a perfect match, is arguably more valuable than a
perfect one, as it constrains the model and guides its next iteration.

\begin{center}\rule{0.5\linewidth}{0.5pt}\end{center}

\subsection{Overall Conclusion}\label{overall-conclusion}

The Theory of the Question of Existence (TQE) proposes that the
emergence of stable physical laws is not a given, but the outcome of a
dual fine-tuning mechanism involving both Energy (E) and Information
(I). The large-scale simulations conducted in this study support this
view, while also highlighting the specific roles of each parameter.

A complementary cohort study (see \textbf{Comparative Analysis: E+I
vs.~E-Only Universes}) further supports this mechanism: energy alone can
attain stability, but adding information decouples complexity from sheer
energetic scale and regularizes extreme anomalies, reinforcing the
two-factor selection picture.

\subsubsection{1. Energy Alone vs.~Energy +
Information}\label{energy-alone-vs.-energy-information}

Universes driven only by Energy can reach stability, but such stability
tends to be rigid, premature, and lacking in genuine complexity. The
introduction of Information enables complexity to emerge as an
independent property, decoupled from pure energetic scale, and produces
universes that are dynamically richer and structurally more realistic.

\subsubsection{2. Dual Selection
Mechanism}\label{dual-selection-mechanism}

The results reveal a two-step process of cosmic selection:

\begin{itemize}
\tightlist
\item
  \textbf{Stability} is gated by the product term \texttt{E·I}, which
  defines a narrow Goldilocks Zone of viable complexity.
\item
  \textbf{Finality} is triggered by the asymmetry
  \texttt{\textbar{}E–I\textbar{}}, with decisive imbalances strongly
  increasing the probability of rapid and permanent Law Lock-in.
\end{itemize}

This resolves the apparent paradox that the ``best'' universes occur at
sub-peak complexity values (X ≈ 12--15) rather than at the maximum
stability region (X ≈ 25.6). The most successful outcomes are not those
with the highest a priori stability probability, but those that are both
stable enough to survive and asymmetric enough to lock in rapidly.

\subsubsection{3. Empirical Predictions and
Discrepancies}\label{empirical-predictions-and-discrepancies}

The model successfully generates rare large-scale anomalies analogous to
the Axis of Evil and the CMB Cold Spot, both emerging exclusively in the
same small subset of early lock-in universes. Strikingly, one simulated
alignment reproduces the observed ≈20° quadrupole--octupole correlation.
However, the cold spot anomalies are consistently more extreme (z ≈
--78) than the Planck measurement (z ≈ --70). This quantitative mismatch
provides a concrete direction for model refinement, rather than
invalidation.

\subsubsection{4. Interpretation and
Limits}\label{interpretation-and-limits}

The findings strongly support the view that Information is not a
peripheral modifier but a fundamental driver of complexity and order. At
the same time, these results are limited to a simulation-based
framework. While the internal logic is consistent and the predictions
falsifiable, empirical validation through cosmological data remains
essential. AI-assisted reasoning checks provided useful consistency
control, but cannot substitute for independent scientific verification.

In summary, the TQE framework demonstrates that stable, law-governed
universes emerge only through a dual selection involving both balance
(\texttt{E·I}) and imbalance (\texttt{\textbar{}E–I\textbar{}}). Energy
alone can sustain existence, but Information transforms existence into
complexity. The model makes concrete, testable predictions, and while
some require refinement, the framework opens a promising pathway toward
an information-theoretic account of cosmogenesis.

\begin{center}\rule{0.5\linewidth}{0.5pt}\end{center}

\subsection{Comparative Analysis: E+I vs.~E-Only
Universes}\label{comparative-analysis-ei-vs.-e-only-universes}

\textbf{A dual-cohort evaluation of stability, complexity, entropy, and
anomalies}

\textbf{Author's Note}\textbar{} All raw data, simulation outputs, and
extended figures from both the E+I and E-only universes are available on
the project's GitHub repository. This includes every simulated universe
across all runs, ensuring full transparency and reproducibility.

This document provides a scientific analysis of simulation results
comparing two distinct types of universes:

\begin{itemize}
\item
  \textbf{E+I Universes: Driven by both Energy (E) and Information (I).}
\item
  \textbf{E-Only Universes: Driven solely by Energy (E).}
\end{itemize}

The primary objective is to evaluate the hypothesis that Information (I)
is a fundamental component necessary for the emergence of a realistic,
complex, and stable universe. The analysis focuses on stability
dynamics, complexity, entropy, cosmological anomalies, and the overall
predictability of the systems.

\subsubsection{Methodological Note}\label{methodological-note}

I computed the summarized analysis based on the \textbf{average results
of 5 E+I simulations and 5 E-Only simulations, which I processed and
compared using Wolfram Language. In these simulations}, I generated each
individual universe through \textbf{10,000 Monte Carlo} iterations to
ensure the statistical robustness of the observed metrics and
differences.

\subsection{Key Findings and
Interpretation}\label{key-findings-and-interpretation}

\subsubsection{1. Stability and Lock-in
Dynamics}\label{stability-and-lock-in-dynamics}

\begin{itemize}
\item
  Observation: E-Only universes demonstrate a higher propensity for
  stabilization (\texttt{stable\_ratio}: 0.518 vs.~0.455) and
  ``lock-in'' (\texttt{lockin\_ratio}: 0.230 vs.~0.186). This indicates
  a stronger tendency to converge into a static, final state. The time
  required to reach stabilization or lock-in is nearly identical in both
  models.
\item
  Interpretation: The presence of Information reduces the tendency for
  premature systemic convergence. While E-Only universes rapidly settle
  into equilibrium---potentially primitive---states, E+I universes
  persist in a dynamic and exploratory phase for a longer duration.
  Information thus functions as a mechanism to sustain dynamism,
  preventing the system from freezing into a simple, static
  configuration too early.
\end{itemize}

\subsubsection{2. Complexity and Entropy}\label{complexity-and-entropy}

\begin{itemize}
\item
  Observation: The most critical finding lies in the relationship
  between Complexity (X) and Energy (E). In E-Only universes, complexity
  is a direct derivative of energy, as evidenced by the identical values
  for \texttt{logX} and \texttt{logE} (2.50). In stark contrast, in E+I
  universes, complexity decouples from energy (\texttt{logX}: 1.30
  vs.~\texttt{logE}: 2.50).
\item
  Interpretation: This evidence strongly suggests that Information
  enables complexity to arise as an independent, emergent property. In
  the E-Only paradigm, complexity is merely a byproduct of the system's
  energy. In the E+I paradigm, Information facilitates a qualitative
  leap, allowing a more sophisticated form of complexity to emerge from
  the system's internal structure and processing capabilities, not just
  its energy content.
\end{itemize}

\subsubsection{3. Cosmological Anomalies}\label{cosmological-anomalies}

\begin{itemize}
\item
  \textbf{Observation}: The analysis of cosmological anomalies provides
  a nuanced but critical insight. As established in the primary analysis
  of the E+I cohort (Figure 23), the simulated Cold Spot anomalies
  (\texttt{cold\_min\_z} mean: -78.8) are consistently more extreme than
  the observed value in our own cosmos (z ≈ -70).

  However, when we isolate the specific role of the Information (I)
  parameter by comparing these results to the E-Only cohort, a clear
  trend emerges. Information exerts a distinct \textbf{regularizing and
  tempering effect}. The E-Only universes, driven by pure energy
  fluctuations, generate even more extreme anomalies
  (\texttt{cold\_min\_z} mean: -83.9).
\item
  \textbf{Interpretation}: These two findings, taken together, resolve
  the apparent contradiction and highlight the precise function of
  Information. The \texttt{I} parameter is crucial for fine-tuning the
  cosmic structure by \textbf{moderating the extreme randomness inherent
  in a purely energy-driven cosmogenesis}. It does not eliminate
  anomalies but tames them, pushing their statistical values closer to a
  realistic regime.

  Therefore, the model correctly identifies Information as a key
  regulating agent. The fact that the E+I model still ``overshoots'' the
  observed data points not to a flaw in the hypothesis, but to a clear
  direction for future model refinement and calibration.
\end{itemize}

\subsection{Overall Theoretical
Conclusion}\label{overall-theoretical-conclusion}

\textbf{Is Information (I) necessary for a realistic, complex, and
stable universe?}

The data provides a multi-faceted answer:

\begin{itemize}
\item
  For \textbf{stability}, Information is not strictly necessary. In
  fact, E-Only universes are more prone to a rigid, static form of
  stability. Information promotes a more dynamic and resilient
  meta-stability by preventing premature lock-in.
\item
  For \textbf{complexity}, the answer is a definitive \textbf{yes}. The
  evidence strongly supports the hypothesis that Information is
  indispensable for the emergence of true, decoupled complexity. Without
  it, complexity remains a mere shadow of energy.
\end{itemize}

Based on these findings, this analysis concludes that while a universe
can exist on the basis of energy alone, the introduction of
\textbf{Information represents a critical phase transition}. This
transition enables the development of emergent complexity, leading to a
more structured, predictable, and dynamically evolving cosmos. The data
validates the thesis that Information is a fundamental, not peripheral,
driver of cosmic evolution.

\begin{center}\rule{0.5\linewidth}{0.5pt}\end{center}

\subsubsection{Methodological Note on Analytical Rigor and
Validation}\label{methodological-note-on-analytical-rigor-and-validation}

The primary conclusions are derived from direct statistical analysis of
the simulation outputs, including the aggregate statistics in
\texttt{summary\_full.json}, the consistency checks in
\texttt{math\_check.json}, the raw run table \texttt{tqe\_runs\_E+I.csv}
(and, where applicable, \texttt{tqe\_runs\_E-Only.csv}), and the full
set of generated figures.

To ensure the utmost rigor in the interpretation of these findings, the
complete dataset and all generated figures were subjected to
\textbf{multiple, independent rounds of control analysis}. The same
simulation data was submitted for a full, iterative, Socratic analysis
to different advanced Large Language Models to act as scientific
reasoning assistants.

Crucially, the key findings, interpretations, and the logical narrative
connecting them \textbf{remained consistent across all independent
analytical rounds}. This process of repeated validation, where different
systems converged on the same conclusions based on the verified, factual
data, provides a high degree of confidence in the robustness of the
results presented here.

In addition to this validated direct analysis, an extensive suite of
predictive machine learning models was developed (the XAI module). While
functional, these models exhibited significant limitations, including
overfitting and internally inconsistent explanations (as shown by
conflicting SHAP and LIME results). For this reason, the conclusions
from these predictive models are considered preliminary and have been
excluded from the main findings of this initial publication,
representing an area for future research. \textbf{However, the
consistently positive signal in metrics such as the \texttt{r2\_delta}
suggests that quantifying the precise impact of Information on the
system's overall predictability remains a promising avenue for future
investigation once these methodological issues are resolved.}

\begin{center}\rule{0.5\linewidth}{0.5pt}\end{center}

\subsection{Discussion}\label{discussion}

\subsubsection{The Main Finding: A Two-Factor Selection
Mechanism}\label{the-main-finding-a-two-factor-selection-mechanism}

The simulation results demonstrate that the formation of stable, ordered
universes is governed not by a single determinant, but by a
\textbf{two-factor selection mechanism}. This dual process resolves the
apparent paradox of why the most successful universes do not arise from
the parameter space with the highest a priori probability of stability.

\textbf{Factor 1: The Stability Gate (E·I).}\\
The analysis confirms the existence of a ``Goldilocks Zone'' defined by
the Complexity parameter \(X = E \cdot I\). This zone functions as a
stability gate: a universe must possess the correct range of complexity,
determined jointly by Energy and Information, to avoid chaos and achieve
stability. Universes with complexity values that are too low or too high
fail to pass through this gate.

\textbf{Factor 2: The Finality Trigger (\textbar E−I\textbar).}\\
Stability alone is not sufficient for the crystallization of immutable
laws. The simulations show that ``Law Lock-in'' is driven by the
Energy--Information asymmetry, quantified as \(|E - I|\). The larger
this asymmetry, the higher the probability that a universe's laws will
rapidly and irreversibly fix. This mechanism acts as a finality trigger,
selecting the winners among already stable candidates.

\textbf{Synthesis.}\\
The most successful universes therefore represent an optimal compromise:
they are complex enough to pass the stability gate, yet asymmetric
enough to activate the finality trigger quickly.

\begin{center}\rule{0.5\linewidth}{0.5pt}\end{center}

\subsubsection{Implications: The TQE Answer to
``Why?''}\label{implications-the-tqe-answer-to-why}

The central research question was: \textbf{Why do stable,
complexity-permitting physical laws exist at all?}\\
The Theory of the Question of Existence (TQE) offers a mechanistic
answer: stable laws are not preordained nor purely random, but the
outcomes of a dynamic selection process during cosmogenesis. Laws of
physics do not descend from a Platonic realm; they \textbf{crystallize
from the fundamental interaction of Energy and Information}.

This distinguishes TQE from prior explanatory frameworks:

\begin{itemize}
\tightlist
\item
  \textbf{Anthropic Principle.} Offers a retrospective justification
  (laws are as they are because we exist to observe them). TQE instead
  provides a predictive, mechanistic account.
\item
  \textbf{Multiverse/String Landscape.} Rather than relying on a cosmic
  lottery across infinite universes, TQE proposes a dynamic evolutionary
  pathway where a single universe can self-organize into stability.
\item
  \textbf{Wheeler's ``It from Bit.''} TQE grounds the conceptual idea of
  information as fundamental in a concrete mathematical and
  computational model, yielding falsifiable predictions (e.g., CMB
  anomalies).
\end{itemize}

\begin{center}\rule{0.5\linewidth}{0.5pt}\end{center}

\subsubsection{Conclusion}\label{conclusion}

The TQE framework reframes the existential question: not \emph{``why
something exists instead of nothing''} but rather \emph{``how order
emerges from chaos.''}\\
The results provide a quantifiable, information-theoretic mechanism
showing that physical law emerges as the crystallized outcome of
energy--information interplay, guided by a two-factor selection process
of stability and asymmetry.

\begin{center}\rule{0.5\linewidth}{0.5pt}\end{center}

\subsection{Limitations of the Model}\label{limitations-of-the-model}

The TQE Framework should be regarded as a stochastic research prototype
rather than a fully developed cosmological simulator. While the
simulation demonstrates the feasibility of modeling emergent laws of
physics through energy--information dynamics, it also carries several
limitations that define clear directions for future research.

\subsubsection{Simulation Framework and Simplified
Physics}\label{simulation-framework-and-simplified-physics}

\begin{itemize}
\tightlist
\item
  \textbf{Abstract Physics}: The model does not simulate physics from
  first principles. Variables representing ``physical laws'' (e.g., A,
  ns, H) are abstract, time is discretized into epochs, and there is no
  explicit representation of space.\\
\item
  \textbf{Heuristic Anomalies}: Implementations of anomalies (e.g., Cold
  Spot, multipole alignment) and fine-tuning diagnostics use simplified
  or heuristic metrics. For example, the CMB maps are synthetic,
  generated from a power-law spectrum (CMB\_POWER\_SLOPE), not from a
  physical plasma simulation of the early universe.
\end{itemize}

\subsubsection{Limits of Predictive Power and
Predictability}\label{limits-of-predictive-power-and-predictability}

The XAI analysis highlighted the complex, non-linear behavior of the
system, revealing predictive limitations: - \textbf{Final State
Classification}: Predicting stable vs.~unstable universes achieves
moderate success (AUC ≈ 0.65).\\
- \textbf{Timing of Lock-in}: Initial conditions (E, I, X) alone are
insufficient to predict the exact timing of ``Law Lock-in'' (R² ≈
0.05).\\
- \textbf{Second-Order Effects}: The model performs poorly in predicting
second-order metrics, such as the delta gain from including the I
parameter (R² \textless{} 0).

This indicates strong \textbf{path-dependence}: outcomes are shaped by
the full stochastic evolution rather than initial parameters alone.

\subsubsection{The ``Cold Spot'' Anomaly and Parameter
Dependence}\label{the-cold-spot-anomaly-and-parameter-dependence}

The Cold Spot anomaly illustrates the model's sensitivity to parameter
choices: - The simulation can generate Cold Spots, but under current
MASTER\_CTRL settings, their magnitudes are ``overshot,'' producing
anomalies stronger than observed.\\
- This highlights parameter dependence (E\_c, σ, α, and other
noise/dynamics constants).\\
- These parameters act as \emph{dials} that can be tuned and calibrated
against empirical data in future refinements.

\begin{center}\rule{0.5\linewidth}{0.5pt}\end{center}

\textbf{Summary}: These limitations do not undermine the framework but
define its current scope: a prototype for exploring emergent cosmic
laws. They also highlight future research directions, particularly
calibration against observational cosmology and refinement of predictive
modules.

\begin{center}\rule{0.5\linewidth}{0.5pt}\end{center}

\subsection{Future Work}\label{future-work}

The TQE Framework in its current form serves as a solid foundation for
several promising avenues of future research. The most important of
these are:

\subsubsection{Model Calibration and
Refinement}\label{model-calibration-and-refinement}

The quantitative discrepancy observed in the ``Cold Spot'' analysis (the
overshooting) does not signify a failure of the model but provides a
clear path for refinement. The next step is a systematic calibration of
the model's parameters---such as the information-coupling strength (α),
the noise models, or the functions governing quantum collapse---against
observational data (e.g., from the Planck satellite).

\subsubsection{Detailed Cosmological
Analysis}\label{detailed-cosmological-analysis}

The present work was limited to a preliminary, heuristic analysis of the
simulated CMB maps. A crucial next step is a full statistical analysis
of the generated ensemble of universes, including the calculation of the
angular power spectrum for each simulated CMB map. This would enable a
rigorous, quantitative comparison of the model's predictions with real
cosmological data, providing a deeper test of the TQE theory's
plausibility.

\subsubsection{Deepening the Theory of the Information
Parameter}\label{deepening-the-theory-of-the-information-parameter}

In its current form, the model treats the Information (I) parameter as a
phenomenological, information-theoretic quantity. Future research should
investigate whether this parameter could be linked to quantum
entanglement, the holographic principle, or other fundamental properties
of a pre-physical state. Developing the theoretical foundation of I is
essential for grounding the TQE framework in fundamental physics.

\begin{center}\rule{0.5\linewidth}{0.5pt}\end{center}

\subsection{Conclusion}\label{conclusion-1}

One of the most profound questions in modern cosmology is why stable
physical laws that permit the emergence of complexity exist at all. The
Theory of the Question of Existence (TQE) offers a novel,
mechanism-based answer, centered on the hypothesis that stable laws
emerge from the coupling of Energy (E) and a fundamental Information (I)
parameter.

The large-scale numerical simulations presented here validate the
internal logic and key predictions of the theory. The main result is the
identification of a \textbf{two-factor selection mechanism}: balance
(E·I) is required for stability, while asymmetry (\textbar E−I\textbar)
is necessary for the finalization of laws (``Lock-in''). This dual
mechanism explains how ``successful'' universes can be dynamically
selected from a chaotic initial state.

The primary contribution of the TQE framework is to move the question of
the origin of physical laws from the realm of philosophical speculation
to that of \textbf{quantitative, falsifiable computational physics}. The
model offers a testable mechanism whose predictions, such as the
statistics of CMB anomalies, can be directly compared with observational
cosmology in future research.

\begin{center}\rule{0.5\linewidth}{0.5pt}\end{center}

\subsection{License}\label{license}

This project is licensed under the MIT License -- see the
\href{./LICENSE}{LICENSE} file for details.

\subsection{Contact}\label{contact}

Got questions, ideas, or feedback?\\
Drop me an email at \textbf{tqe.simulation@gmail.com}
